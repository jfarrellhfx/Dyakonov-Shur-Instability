\documentclass[12pt]{article}
\usepackage[utf8]{inputenc}
\usepackage[margin=1in]{geometry}
\usepackage{float}
\usepackage{graphicx}
\usepackage{physics}
\usepackage{amsmath}
\usepackage{amssymb}
\usepackage[colorlinks]{hyperref}
\usepackage{subcaption}
\captionsetup{font=footnotesize}
\usepackage{booktabs}
\usepackage{multirow}
\usepackage{newtxtext, newtxmath}
\newcommand{\otoprule}{\midrule[\heavyrulewidth]}
\usepackage[export]{adjustbox}

\title{Linearized Equations and BCs}
\author{Jack Farrell}

\begin{document}
	\maketitle
	We're studying the equations
	
	\begin{equation}
	\label{eq:mainEquation}
	\begin{aligned}[c]
	\pdv{J}{t} + \pdv{r}\left(\frac{J^2}{n} + v_s^2 n \right) -\frac{\eta}{m} \left(\pdv[2]{r} + \frac{1}{r}\pdv{r}\right)\frac{J}{n} = \gamma (J_0 - J) - \frac{J^2}{n} \frac{1}{r},  \\
	\pdv{n}{t} + \pdv{J}{r} = -\frac{J}{r}. \\
	\end{aligned}
	\end{equation}
	
	We are going to try to linearize these by writing $n = n_0 + n_1 e^{-\mathrm{i} \omega t}$ and $J = J_0 + J_1 e^{-\mathrm{i} \omega t}$.  We'll also git rid of any terms quadratic in any of the parameters $\eta$, $\gamma$, and $v_0$. So we'll only get the frequency $\omega$ accurately to that same order.
	
	\section{Boundary Conditions}
	Remember that the steady state is $J_0(r) = n_0v_0 R_2 / r$.  That makes the Neumann condition a little tricky.  We want the BCs:
	\begin{equation*}
	\begin{aligned}
	\pdv{J}{r} \left(r = R_1\right) &= 0 \\
	J\left(r = R_2\right) &= n_0 v_0
	\end{aligned}
	\end{equation*}
	But if $J = J_0 + J_1e^{-\mathrm{i} \omega t}$ and $n = n_0 + n_1e^{-\mathrm{i} \omega t}$, then for the Neumann condition, you need, at $r = R_2$, the derivative to vanish:
	\[ \pdv{J_0}{r} + \pdv{J_1}{r}e^{i \omega t} = 0.\]
	But the steady state solution $J_0$ does \textit{not} obey this Neumann condition.  So I think the boundary conditions are (in terms of the linearized  things, $n_1$ and $J_1$):
	\begin{equation}
	\begin{aligned}
	\pdv{J}{r} \left(r = R_1\right) &= \frac{n_0 v_0 R_2}{R_1^2}e^{\mathrm{i} \omega t} \\
	J\left(r = R_2\right) &= 0
	\end{aligned}
	\end{equation}
	(NB: if $R_1$ is very large compared to the length scale of the problem, which is $L$, the Neumann condition looks pretty much homogenous). 
	
	\section{Equations}
	For the real part of the frequency, you get:
	\begin{equation}
	r^2\pdv[2]{J_1}{r} + r\pdv{J_1}{r} + \left(r^2\frac{\omega_1^2}{v_s^2} - 1 \right)J_1 = 0.
	\end{equation}
	For the complex part of the frequency, you get:
	\begin{equation}
	r\pdv[2]{J_1}{r} + \left(1 - \frac{2v_0R_2}{\nu}\right)\pdv{J_1}{r} - \frac{1}{\nu}r(2\omega_2 + \gamma)J_1=0
	\end{equation}
\end{document}